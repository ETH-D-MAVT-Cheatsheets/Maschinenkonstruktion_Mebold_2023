\subsection{schaltbar \hfill ME}
\subsubsection{Fliehkraftkupplung \hfill ME}
\begin{footnotesize}
    \begin{center}
        \scriptsize{\textbf{ohne Abstand zw. Fliehgewicht und Trommel:}}
        \begin{empheq}[box=\fbox]{align*}
            F_F = F_Z \quad \mid \quad \omega = 2\pi \cdot n
            \\ F_F = F_v \cdot x \quad \mid \quad F_Z = m\cdot r_s \omega^2 \quad \mid \quad \text{\textcolor{Red}{EINHEITEN!}}
            \\ n_G > \sqrt{\frac{F_F}{4 \cdot \pi^2 \cdot m \cdot r_s}} \: [s^{-1}] \xrightarrow{\cdot 60} [min^{-1}]
        \end{empheq}
        \scriptsize{\textbf{mit Auslenkung $a$ des Fliehgewichtes:}}
        \begin{empheq}[box=\fbox]{align*}
            F_F = F_Z \quad \mid \quad \omega = 2\pi \cdot n
            \\F_F = x \cdot(F_v + 2 \cdot R \cdot a) \quad \mid \quad F_Z = m \cdot (r_s + a) \omega^2
            \\n_E > \sqrt{\frac{F_F}{4 \cdot \pi^2 \cdot m \cdot (r_s + a)}} \: [s^{-1}] \xrightarrow{\cdot 60} [min^{-1}]
        \end{empheq}
        \scriptsize{\textbf{Übertragung $M_{\text{max}}$ mit $\mu$ und Auslenkung $a$ des Fliehgewichtes:}}
        \begin{empheq}[box=\fbox]{align*}
            M_{\text{max}} = F_R \cdot r_k = y \cdot \mu \cdot F_N \cdot r_k \quad \mid \quad F_N = \frac{M_{max}}{y \mu r_k}
            \\F_Z = F_N + F_{F} \quad \mid \quad \omega = 2\pi \cdot n
            \\F_F = x \cdot(F_v + 2 \cdot R \cdot a) \quad \mid \quad F_Z = m \cdot (r_s + a) \omega^2
            \\n_B = \sqrt{\frac{F_F + F_N}{4 \cdot \pi^2 \cdot m(r_s + a)}} \: [s^{-1}] \xrightarrow{\cdot 60} [min^{-1}]
        \end{empheq}
        \begin{scriptsize}
            $F_Z = \text{Zentrifugalkraft}; \; F_v = \text{Vorspannkraft}; \; m = \text{Masse Fliehgewicht} [kg];$
            \\ $r_s = \text{Radius Schwerpunkt} [m]; \; R = \text{Federrate pro Feder}; \; x = \text{Anzahl Federn};$
            \\ $n_G = \text{Grenzdrehzahl}; \; n_E = \text{Einschaltdrehzahl}; \; \mu = \text{Reibwert}$;
            \\ $r_k = \text{Innenradius Trommel}; \; a = \text{Abstand zwischen Fliehgewicht und Trommel} [m]$
            \\ $F_F = \text{Federkraft (aller Federn)}; \; y = \text{Anzahl Fliehgewichte}$
        \\~\\ \textcolor{Red}{ACHTUNG: $n_X = [s^{-1}]$}
        \end{scriptsize}
    \end{center}
\end{footnotesize}

\subsubsection{Rutschkupplung \hfill ME}
\begin{footnotesize}
    \begin{center}
        \begin{empheq}[box=\fbox]{align*}
            M_G &= F_t \cdot r_m = F_R \cdot r_m = i \mu_G \cdot F_N \cdot r_m 
            \\&= i \mu_G \cdot F_v \cdot r_m 
            \\r_m &= \frac{d_a + d_i}{4} \quad \mid \quad p = \frac{F_v}{A_m} \quad \mid \quad A_m = \frac{\pi}{4}(d_a^2-d_i^2)
            \\ &\textbf{wenn zusätzlicher Weg $\Delta s$:} 
            \\ F_{v, \text{neu}} &= F_v + R_{\text{ers} \cdot \Delta s} \Rightarrow M_G = i \cdot \mu_G \cdot F_{v, \text{neu}} \cdot r_m
        \end{empheq}
        \begin{scriptsize}
            $M_G = \text{Grenzdrehmoment}; \; F_N = \text{Normalkraft}; \; F_R = \text{Reibkraft}; \;$
            \\$\mu_G = \text{Gleitreibungskoeffizient}; \; r_m = \text{mittlerer Scheibenrand}; \; i = \text{Anzahl Reibkontakte}
        $
        \end{scriptsize}
    \end{center}
\end{footnotesize}