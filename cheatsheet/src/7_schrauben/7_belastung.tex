\subsection{Belastung Schraube \hfill ME}
\begin{footnotesize}
        \begin{center}
            \begin{empheq}[box=\fbox]{align*}
                \sigma_z &= \frac{F_{s, \text{max}}}{A_s} \quad \mid \quad \tau_t = \frac{M_{t, \text{max}}}{W_{p,3}} \quad \mid \quad \tan(\rho) = \frac{\mu_G}{cos(\frac{\beta}{2})}
                \\M_A &= M_G + M_K
                \\M_G &= F_V \frac{d_2}{2} \tan(\psi + \rho_G') \quad \mid \quad M_K = F_V \mu_K\frac{D_{km}}{2}
            \end{empheq}
            \begin{empheq}[box=\fbox]{align*}
                \text{Für metrische}\: \text{Regelgewinde} \:&\text{mit mittleren Reibungszahlen:}
                \\M_A \approx 0.17 &\cdot F_V \cdot d
            \end{empheq}
        \end{center}
        \begin{scriptsize}
            \begin{center}
                $F_{s, \text{max}}$ = Schraubenkraft; $A_s$ = Spannungsquerschnitt;
                \\$M_{M, \text{max}}$ = Montagemoment; $M_A$ = Anziehmoment; $\sigma_z$ = Zugspannung;
                \\$F_V$ = Vorspannkraft; $\rho_G'$ = Reibungswinkel; $\tau_t$ = Torsionsspannung
        \end{center}
        \end{scriptsize}
\end{footnotesize}