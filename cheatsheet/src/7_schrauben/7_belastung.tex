\subsection{Belastung Schraube \hfill ME}
\begin{footnotesize}
    \begin{center}
        \begin{empheq}[box=\fbox]{align*}
            \sigma_z &= \frac{F_{s, \text{max}}}{A_s} \quad \mid \quad \tau_t = \frac{M_{t, \text{max}}}{W_{t,3}} \quad \mid \quad W_{t,3} = \frac{\pi d_3^3}{16}
            \\M_A &= M_G + M_K \quad \mid \quad \tan(\rho) = \frac{\mu_G}{cos(\frac{\beta}{2})}
            \\M_{G,\text{max}} &= F_{\textrm{VM}} \frac{d_2}{2} \tan(\varphi + \rho) \quad \mid \quad M_K = F_{\textrm{VM}} \mu_K\frac{D_{km}}{2}
        \end{empheq}
        \begin{empheq}[box=\fbox]{align*}
            \text{Für metrische}\: \text{Regelgewinde} \:&\text{mit mittleren Reibungszahlen:}
            \\M_A \approx 0.17 &\cdot F_{\textrm{VM}} \cdot d
        \end{empheq}
    \end{center}
    \begin{scriptsize}
        \begin{center}
            $F_{s, \text{max}}$ = Schraubenkraft; $A_s$ = Spannungsquerschnitt;
            \\$M_{M, \text{max}}$ = Montagemoment; $M_A$ = Anziehmoment; $\sigma_z$ = Zugspannung;
            \\$F_{\textrm{VM}}$ = Vorspannkraft; $\rho$ = Reibungswinkel; $\tau_t$ = Torsionsspannung
    \end{center}
    \end{scriptsize}
\end{footnotesize}