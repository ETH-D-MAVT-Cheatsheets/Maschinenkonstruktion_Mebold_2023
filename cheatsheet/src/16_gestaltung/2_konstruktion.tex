\subsection{Konstruktion \hfill IP}
    \subsubsection{Package \hfill IP}
        \begin{scriptsize}
            \begin{center}
                \begin{itemize}
                    \item \textbf{Package:} dient zur Aufteilung von Arbeitspaketen in der Konstruktion / Definition von Schnittstellen und Abmessungen / Kollisionen, Montierbarkeit / Einhaltung von Mindestabständen
                \end{itemize}
            \end{center}
        \end{scriptsize}
    
    \subsubsection{Grundregeln der Gestaltung \hfill IP}
    \begin{scriptsize}
        \begin{center}
            \begin{itemize}
                \item \textbf{Einfach $\to$} weniger potentiele Fehlerquellen / geringere Kosten
                \item \textbf{Eindeutig $\to$} einfachere Berechnung / zuverlässigere Funktionserfüllung
                \item \textbf{Sicher $\to$} weniger Unfälle
            \end{itemize}
        \end{center}
    \end{scriptsize}

    \subsubsection{Sicherheitstechniken \hfill IP}
    \begin{footnotesize}
        \begin{center}
            \begin{tabular}{|c|c|}
                \hline
                \cellcolor{Red} hinweisend & auf Gefahr hinweisen \\
                \hline
                \cellcolor{Yellow} mittelbar & Mittel zum Schutz \\
                \hline
                \cellcolor{Green} unmittelbar & Gefahr vermeiden\\
                \hline
            \end{tabular}
        \end{center}
    \end{footnotesize}

    \subsubsection{Prinzip der abgestimmten Verformung \hfill IP}
    \begin{scriptsize}
        \begin{itemize}
            \item \textbf{Ziel:} gleichmässige Werlstoffausnutzung $\Rightarrow$ gleichgerichtete Verformung / \\möglichst keine Relativverformung (Bsp. PKW Welle)
        \end{itemize}
    \end{scriptsize}

    \subsubsection{Prinzip der Selbsthilfe \hfill IP}
    \begin{scriptsize}
        \begin{itemize}
            \item \textbf{Ziel:} gewillte Wirkung wird durch Beanspruchung erhöht / schützender Effekt bei \\Überbeanspruchung $\Rightarrow$ Kraftleitung der Störwirkung in Richtung gewollter Kraft-\\richtung / veränderte Kraftleitung bei Überlast / Störwirkung in dieselbe Richtung \\wie Nutzwirkung (Bsp. Seilknoten) 
         \end{itemize}
    \end{scriptsize}
