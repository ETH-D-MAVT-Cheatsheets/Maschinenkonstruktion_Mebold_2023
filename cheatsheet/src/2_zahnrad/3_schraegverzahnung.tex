\subsection{Schrägverzahnte Stirnräder \hfill ME}
\begin{scriptsize}
    \begin{itemize}
    \item Ermöglicht Verlängerung d. Eingriffsstrecke ($\Delta q$) $\to$ höhere Belastbarkeit \\und ruhigeres Laufverhalten $\to$ teuer
    \end{itemize}
\end{scriptsize}
\vspace{-3mm}
\begin{minipage}{0.58\linewidth}
    \begin{footnotesize}
        \begin{center}
            \begin{empheq}[box=\fbox]{align*}
                \Delta q &= b \cdot tan(\beta)
                \\m_t &= \frac{m_n}{cos(\beta)}
                \\cos(\beta) &= \frac{m_n}{m_t} = \frac{m_n \cdot z}{d_{tk}}
                \\d &= m_t \cdot z
                \\z'_{gt} &= 14 \cdot cos^3(\beta)
            \end{empheq}
        \end{center}
    \end{footnotesize}
\end{minipage}
\begin{minipage}{0.4\linewidth}
    \begin{scriptsize}
            \begin{align*}
                b &= \text{Breite ZR}
                \\m_t &= \text{Stirnmodul}
                \\m_n &= \text{Normalmodul}
                \\z'_{gt} &= \text{Grenzzähnezahl}
                \\ d_f &= \text{Durchmesser Fuge}
                \\ E_A &= \text{E-Modul Nabe}
                \\K &= \text{aus Tabelle}
                \\ &\textbf{Grenzzähnezahl:}
                \\ &\text{Theoretisch:} \; z < 17
                \\ &\text{Praktisch:} \; z < 14
            \end{align*}
    \end{scriptsize}
\end{minipage}

\subsubsection{Kräfte - schrägverzahntes Zahnrad \hfill ME}
\vspace{-1mm}\begin{minipage}{0.4\linewidth}
    \begin{footnotesize}
        \begin{center}
            \mathbox{
                F_{t1} = F_{t2} = \frac{2M_1}{d_1}
            }
        \end{center}
    \end{footnotesize}
\end{minipage}
\begin{minipage}{0.58\linewidth}
    \begin{footnotesize}
        \begin{center}
          \mathbox{
             F_{r1} = F_{r2} = \frac{F_{t1}}{cos(\beta)} \cdot tan(\alpha)
          }  
          \vspace{-2mm}
          \mathbox{
                F_{a1} = F_{a2} = F_{t1}\cdot tan(\beta)
            }
        \end{center}
    \end{footnotesize}
\end{minipage}