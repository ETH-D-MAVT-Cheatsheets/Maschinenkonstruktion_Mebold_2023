\begin{footnotesize}
    \mathbox{
        \sigma_b = \frac{M_b}{W_b} \quad \mid \quad \tau_t = \frac{M_t}{W_t}
    }
    \begin{empheq}[box=\fbox]{align*}
        \text{\scriptsize\underline{nur Biegung: } } d_{\text{min}} > \sqrt[3]{\frac{16M_t}{\pi \tau_{z, \text{zul}}}} \quad \mid \quad \text{\scriptsize\underline{nur Torsion: } } d_{\text{min}} > \sqrt[3]{\frac{16 M_t}{\pi \cdot \tau_{\text{zul}}}}
    \end{empheq}
    \cbreak

    \begin{empheq}[box=\fbox]{align*}
        \underline{\text{Vollwelle:}} \; W_b = \frac{\pi \cdot d^3}{32} \quad &\mid \quad W_t = \frac{\pi \cdot d^3}{16}    
        \\\underline{\text{Hohlwelle:}} \; W_b = \frac{\pi \cdot (D^4-d^4)}{32 \cdot D} \quad &\mid \quad W_t = \frac{\pi \cdot (D^4-d^4)}{16\cdot D}
    \end{empheq}
    \scriptsize{$\sigma_b$ = Biegespannung; $M_b$ = Biegemoment $[F\cdot r]$; \\$W_b$ od. $W_{ax}$ = axiales Widerstandsmoment; $M_t$ = Torsionsmoment $[F\cdot r]$; \\$W_t$ od. $W_{p}$ = polares Widerstandsmoment}
\end{footnotesize}