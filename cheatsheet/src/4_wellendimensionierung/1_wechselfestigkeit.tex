\subsection{Wechselfestigkeit \hfill ME}
\begin{footnotesize}
    \begin{empheq}[box=\fbox]{align*}
        \sigma_v &= \sqrt{\sigma_b^2 + 3\tau_t^2} \quad \mid \quad S_D = \frac{R}{B} = \frac{\sigma_{b,W}}{\sigma_v} = \frac{\tau_{t,W}}{\tau_{t, \text{zul}}}
       \\ \sigma_v &= \sqrt{(\alpha_b \cdot \sigma_b)^2 + 3\cdot (\alpha_t \cdot \tau_t)^2} \quad \mid \quad \tau_{\text{zul}} = \frac{\tau_{t,w}}{S_D}
       \\ S_D &= 3 \text{ (kurze Welle)};\quad S_D = 4 \text{ (lange Welle)}
    \end{empheq}
    \scriptsize{$\sigma_v$ = Vergleichsspannung; $\sigma_{b,W}$ = Biege-Wechselfestigkeit; \\$\tau_{t,W}$ = Torsions-Wechselfestigkeit; $S_D$ = Sicherheit gegen Bruch}
\end{footnotesize}