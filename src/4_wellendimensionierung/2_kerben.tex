\subsection{Kerben \hfill ME}
\begin{footnotesize}
    \begin{minipage}{0.58\linewidth}
        \begin{center}
            \mathbox{
                \alpha_k = \frac{\sigma_{\text{max}}}{\sigma_n} = \frac{\tau_{\text{max}}}{\tau_n}
            }
        \end{center}
    \end{minipage}
    \begin{minipage}{0.4\linewidth}
        \begin{scriptsize}
            \begin{center}
                $\alpha_k$ = Kerbformzahl
                \\$\beta$ = Kerbwirkungszahl 
                \\$\sigma_n$ = Nennspannung $(= \sigma_b)$
                \\(Zustand ohne Kerbe)
                \\$\tau_n$ = Nenntorsion $(= \tau_t)$
            \end{center}
        \end{scriptsize}
    \end{minipage}
    \begin{itemize}
        \item \scriptsize Bei \underline{Umfangsnuten} (Freistich) und \underline{Wellenabsätzen} immer \textbf{kleineren} Durchmesser nehmen für Nennspannung ($\sigma_n, \tau_n$).
        \item \scriptsize Bei \underline{Passfedernuten} und \underline{Keilwellen} immer \textbf{grösseren} Durchmesser nehmen für Nennspannung ($\sigma_n, \tau_n$)
    \end{itemize}
\end{footnotesize}