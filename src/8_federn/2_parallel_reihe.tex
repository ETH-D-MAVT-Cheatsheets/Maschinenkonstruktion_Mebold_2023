\subsection{Parallel vs. Reihenschaltung \hfill ME}
\begin{footnotesize}
    \begin{center}
        \begin{minipage}{0.5\linewidth}
            \begin{empheq}[box = \fbox]{align*}
                &\text{Parallelschaltung:}
                \\F &= F_1 + F_2 + F_3
                \\s_1 &= s_2 = s_3 = s
                \\R_{\text{ers}} &= R_1 + R_2 + R_3
            \end{empheq}
        \end{minipage}
        \begin{minipage}{0.48\linewidth}
            \begin{empheq}[box= \fbox]{align*}
                &\text{Reihenschaltung:}
                \\s &= s_1+s_2+s_3
                \\F &= F_1+F_2+F_3
                \\R_{\text{ers}} &= \frac{1}{\frac{1}{R_1}+\frac{1}{R_2}+\frac{1}{R_3}}
            \end{empheq}
        \end{minipage}
            \begin{empheq}[box=\fbox]{align*}
                &\text{Kombiniert (zuerst Parallel-, dann Reihenschaltungen):} 
                \\ &R = \frac{1}{\frac{1}{R_1+R_2} + \frac{1}{R_3 + R_4}}
            \end{empheq}
    \end{center}    
\end{footnotesize}