\subsection{Drehfedern \hfill ME}
\begin{footnotesize}
    \begin{empheq}[box=\fbox]{align*}
        M = R \cdot \varphi \quad &\mid \quad W = \frac{1}{2} \cdot R\cdot(\varphi_2^2 - \varphi_1^2)
        \\ \tau_{max} = \frac{T}{W_T} \quad &\mid \quad W_T = \frac{\pi R^3}{2}
        \\ \text{Schenkelfeder:} \quad R &= \frac{\pi}{180^\circ} \cdot \frac{E\cdot d^4}{64 \cdot D \cdot n}
        \\ \text{Drehstabfeder:} \quad R &= \frac{\pi}{180^\circ} \cdot \frac{\pi \cdot G \cdot d^4}{32\cdot l_f}
    \end{empheq}
    \begin{scriptsize}
        $M$ = Moment; $\varphi$ = Verdrehwinkel; $W$ = Arbeit; $E$ = E-Modul; $G$ = G-Modul; \\$n$ = Anzahl federnder Windungen; $l_f$ = best. Länge bei Drehstabfeder (idR gegeben) 
    \end{scriptsize}
\end{footnotesize}